%LuaLateX
%compilen per :!lualatex %

%Standard
\documentclass[12pt, a4paper]{article}
\usepackage{fontspec}
%\usepackage{marginnote}
\usepackage[german]{babel}
\usepackage[hidelinks]{hyperref}

%Grafik
\usepackage{float}
%\usepackage{pgf-pie}
\usepackage{pdfpages}
%\usepackage{pgf, tikz} %Tikz GRAPHICS

%Seitenlayout, Margins verkleinern
\usepackage[a4paper, top=2.5cm,bottom=30mm, left=3cm, right=3cm]{geometry} 
\usepackage{setspace}%Zeilenabstand verändern zB \onehalfspacing(Präambel)

%Mathe, Grafiken
\usepackage{amsmath}
\usepackage{csquotes}
\usepackage{graphicx} %Für Bilder
\usepackage{subcaption} %Für subfigure
\usepackage{siunitx}
\usepackage{parskip}
%Bruchstrich
\sisetup{per-mode=fraction, output-decimal-marker = {,}, exponent-product=\cdot}
%Hier beliebige Einheiten hinzufügen
\DeclareSIUnit\year{Jahre}

%Kopf-Fußzeile
\usepackage{fancyhdr}
%\fancyhf{}
\pagestyle{fancy}
\fancyhead[L]{}
\fancyhead[R]{\includegraphics[width=1.75cm]{kcwlogo.png}}
\renewcommand{\headrulewidth}{0pt}
\cfoot{}
\usepackage{marginnote}

%Section auf größe von subsection setzen
\usepackage{titlesec}
\titleformat*{\section}{\large\bfseries}


%Titel
\title{Spielplan: 2. Kanupoloturnier}
\date{}
\author{}
%Dokument
\begin{document}
\maketitle
\thispagestyle{fancy}
\section*{Die Teams}
    \begin{table}[H]
        \centering
        \begin{tabular}{|
            l|
            
            }
        
            {{ line }}
        
        \end{tabular}
    \end{table}

\section*{ {{ day }} }
    \begin{table}[H]
        \centering
        \begin{tabular}{l|l|l|l|lcl|l|l}
            Nr & Zeit & Gr. & Team A & \multicolumn{3}{|c|}{Ergebnis} & Team B & Schiedsrichter \\\hline\hline
            
            {{ game }}
            
        \end{tabular}
    \end{table}

\end{document}
